
\section{Introduzione} % (fold)
\label{sec:introduction}
    
    La seguente relazione è stata redatta contestualmente allo svolgimento del progetto di \emph{Advanced Scheduling Systems} durante l'anno accademico 2016/2017. Per lo svolgimento del progetto e la stesura di tale documento sono state seguite le specifiche concordate con il docente e responsabile del corso Prof. Andrea Schaerf.

    Il progetto è interamente scritto in C++, e fa utilizzo del framework per la ricerca locale EasyLocal++ (\href{https://bitbucket.org/satt/easylocal-3}{\texttt{https://bitbucket.org/satt/easylocal-3}}).

    I sorgenti sono stati testati sulle seguenti macchine
    \begin{itemize}
        \item[--] Notebook Acer Travelmate 5760 series con processore Intel Core i5 2450M e sistema operativo Arch Linux (Linux 4.6.4-1-ARCH x86\_64);
        \item[--] MacBook Pro, 2,4 GHz Intel Core i5, 8 GB 1600 MHz DDR3
    \end{itemize}
    e con i seguenti compilatori:
    \begin{itemize}
        \item[--] G++ (GCC) 6.3.1 20170109
        \item[--] Clang++ (Clang) 3.9.1
    \end{itemize}

    Tutti i file sorgenti necessari verranno forniti assieme a questa relazione, con relativi \emph{Makefile} per la compilazione. 

% section introduction (end)