
\section{Conclusioni} % (fold)
\label{sec:conclusioni}
	
    La scelta delle cinque mosse si è rivelata una buona strategia sulla base dei risultati ottenuti. Il loro vantaggio principale è stato quello di poter calcolare il $\Delta$-cost in modo efficiente, e questo ha consentito un notevole speed-up nelle performace. Un altro punto di forza è stato l'implementazione dell'\textit{Hungarian Algorithm}, che ha reso possibile l'uso della BestMove nello steepest descent, che altrimenti sarebbe risultato proibitivo usato l'enumerazione completa tramite le funzioni FirstMove e NextMove.
    
    Con la combinazione di queste cinque mosse si è ottenuto un risultato di $418$ \textit{matching edges} ($62$ conflitti) su un totale di $467$: scorrendo la letteratura su questo problema, tale risultato è sufficientemente buono da poter giustificare la scelta di tali mosse.

    Sviluppi futuri di questo progetto potrebbero essere i seguenti:
    \begin{itemize}
    	\item eseguire il \textit{parameter tuning} sul simulated annealing per capire il valore ottimale dei parametri
    	\item usare il solver \textit{Multi Start} di easylocal++, evitando di redirigere in input il file con i comandi
    	\item implementare mosse non-standard per capire se è possibile un effettivo miglioramento
    \end{itemize}

% section conclusioni (end)